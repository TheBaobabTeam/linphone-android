\documentclass[a4paper]{article}
\usepackage{graphicx} %Required for diagrams
\usepackage[bookmarks=true]{hyperref}
\usepackage{bookmark}%Required to do pdf bookmarking
\usepackage[margin=1.2in]{geometry}
\usepackage{float}
\usepackage{caption}
\usepackage{hyperref}%Required for referencing website pages
\usepackage[english]{babel}

\usepackage{graphicx}
\usepackage{dcolumn}
\usepackage[table]{xcolor}

\title{User Manual}
\author{Baobab Team}

\begin{document}
\newpage


\begin{titlepage}

\begin{center}

\includegraphics[width=400px]{pictures/logo.jpg}
\vspace{1 cm}
\begin{flushright} \large
\begin{tabular}{lr}
\vspace{1 cm}
\LARGE\textbf{Document:} Architectural Requirements\\
\vspace{1 cm}
\LARGE\textbf{Project:} Group Chat For Linphone (Agile DO-178)\\
\LARGE\textbf{Date: }\today\\
\end{tabular}
\end{flushright}

\centering \includegraphics[width=350px]{pictures/Team.jpg}

Patience Mtsweni, Lerato Molokomme, Tsepo Ntsaba, Mpedi Mello, Lutfiyya Razak, Ephiphania Munava\\
\vspace{1 cm}
\large\textbf{Github repository: } \\
\url{https://github.com/TheBaobabTeam/linphone-android}

\end{center}
\end{titlepage}
\newpage

\section{Introduction}
This User Manual contains all information that is essential to give the user guidance and support on how to use the Linphone-Android application. It includes a description on system configuration, installation, using the system, troubleshooting, etc. 

\section{General Information}

\subsection{Project Details}

\begin{table}[h]
\setlength{\arrayrulewidth}{0.5mm}
\setlength{\tabcolsep}{14pt}
\renewcommand{\arraystretch}{2} 
\begin{tabular}{ |p{6cm}|p{6cm}|p{6cm}|  }
\hline
\rowcolor{lightgray}\multicolumn{2}{|c|}{\textbf{System name and the names and/or affiliation of all stakeholders}} \\
\hline
System name & Linphone-Android Group Chat \\
\hline
Stakeholder & Kobus Coetzee \\
\hline
Scrum master  & Potego Mello\\ \hline 
Client representative  & Patience Mtsweni\\ \hline 
UX developer  & Lutfiyya Razak and Lerato Molokomme\\ \hline 
Backend developer  & Ephiphania Munava and Potego Mello\\ \hline 
Crypto developer  & Tsepo Ntsaba \\ \hline 
CI / CD support  & Patience Mtsweni \\ 
\hline
\end{tabular}
\caption {}
\end{table}

\subsection{System Overview}
The system we are working on is called Linphone.\\
Linphone is an open source internet SIP phone or Voice Over IP phone(VoIP), compatible with the SIP protocol.  \\
It allows for users to chat instantly or to make calls over the internetThe system we are working on is called Linphone by using the SIP protocol to one another. \\
Linphone has been developed for different platforms but this manual only focuses on Linphone for Android devices.\\
A new feature has been added to this application - the group chat functionality. This allows for 2 or more people to participate in what we call a group chat. Their communication will take place in the group chatroom.

\subsection{System Configuration}

%Ephiphania Edit Here
\subsubsection{Equipment}
An Android phone is a smartphone that runs on Google's open-source Android operating system. There are a number of different manufacturers that make Android phones, namely HUAWEI, LG, ASUS, acer, Virgin mobile, Samsung, Motorola, Sony and many more.\\

\begin{center}
\begin{figure}[h]
\centering
\includegraphics[width=0.7\linewidth]{./pictures/android.jpg}
\caption{\label{fig:Agile}Android Phones.}
\end{figure}
\end{center}

\subsubsection{Network}
The linphone application requires registration to a SIP provider before you can operate it. The provider ensures that you have a SIP address which is a Uniform Resource Identifier also known as a URI which is represented in the form user@domain.tld. \\


\begin{center}
\begin{figure}[h]
\centering
\includegraphics[width=0.7\linewidth]{./pictures/sip.jpg}
\caption{\label{fig:Agile}SIP Protocol.}
\end{figure}
\end{center}
Linphone.org hosts a free SIP service which allows users to make audio or video calls using SIP addresses under the domain sip.linphone.org. \\
When creating your sip address like sip:baobab@sip.linphone.org you simple use the form provided on http://www.linphone.org/free-sip-service.html and your friends can communicate with you by simply making use of the sip address.


\subsection{Installation}
\begin{itemize}
\item Turn on your Android Smartphone and click on the Google Play icon to launch Google Play.
\item Tap the "Search" button in the upper right corner and type "Linphone"
\item From the result list tap the result with the title "Linphone Video" to expand it
\item Tap the "Install" button, Linphone for Android will be downloaded and the installation procedure will begin
\item Confirm that you accept the Linphone application rights on the screen that will be appear on your Smartphone by tapping "Accept". When the file installation is done tap "Open".
\item Press "Agree" when the License Agreement appears on your screen to finalize the install and start Linphone for Android.
\end{itemize}

\newpage

\subsection{Getting started}
%Patience Edit Here
\subsubsection{Registering on the network}
To get started with the system, you need to install Linphone Application onto your android phone. To do this refer to section(2.4), and follow the results as shown there.
Once you have Linphone installed, you should setup your sip account, which you will need to be able to use the system. For more details on what a SIP address is, refer to section (2.3.2).
\begin{itemize}
\item Click on the \textbf{settings} button on the bottom of the screen. Next click on  \textbf{Account Setup Assistant}.
\end{itemize}

\begin{figure}[h]
  \centering
  \begin{minipage}[h]{0.4\textwidth}
    \includegraphics[width=40mm, scale=0.5]{pictures/home.png}
    \caption{Home Screen}
  \end{minipage}
  \hfill
  \begin{minipage}[h]{0.4\textwidth}
    \includegraphics[width=40mm, scale=0.5]{pictures/settings.png}
    \caption{Settings Screen}
  \end{minipage}
\end{figure}

\begin{itemize}
\item Click on \textbf{Let's go}. Next, click on \textbf{Create account on linphone.org} then click \textbf{Create} . \\
\item Otherwise, if you already have a SIP or linphone.org account, click on their options and click \textbf{Apply} .
\end{itemize}

\begin{figure}[h]
  \centering
  \begin{minipage}[h]{0.4\textwidth}
    \includegraphics[width=40mm, scale=0.5]{pictures/welcome.png}
    \caption{Account Setup Screen}
  \end{minipage}
  \hfill
  \begin{minipage}[h]{0.4\textwidth}
    \includegraphics[width=40mm, scale=0.5]{pictures/options.png}
    \caption{Options Screen}
  \end{minipage}
\end{figure}


\newpage

\subsection{Using the system}
%Potego Edit Here

After successful installation and setup (sip account) of the linphone application, you should then see the following screen:

\begin{center}
\begin{figure}[H]
\centering
\includegraphics[width=0.7\linewidth]{pictures/home.png}
\caption{\label{fig:Screen1}Main Screen.}
\end{figure}
\end{center}

To make call via VoIP (Voice over Internet Protocol), type in (where it's written "Number or address") the sip address or the phone number of the person you would like to contact and press the dial button like screen two:

\begin{center}
\begin{figure}[H]
\centering
\includegraphics[width=0.7\linewidth]{pictures/Screenshot_2015-08-04-05-38-27.png}
\caption{\label{fig:Screen2}Screen two.}
\end{figure}
\end{center}

To start chatting just click on the chat button in the menu at bottom of the screen. It will take you to the following screen:

\begin{center}
\begin{figure}[H]
\centering
\includegraphics[width=0.7\linewidth]{pictures/s1.jpg}
\caption{\label{fig:Screen3}Screen three.}
\end{figure}
\end{center}

Now click on the "new conversation" button at the top left position which will then lead you to the your contact list where you have to select the contact you would like to chat with like in the following:

\begin{center}
\begin{figure}[H]
\centering
\includegraphics[width=0.7\linewidth]{pictures/Screenshot_2015-08-04-05-37-18.png}
\caption{\label{fig:Screen4}Contact List.}
\end{figure}
\end{center}

Selecting a contact to chat with should now take you to the chatroom which is the place where messages are exchanged. Look at the following:

\begin{center}
\begin{figure}[H]
\centering
\includegraphics[width=0.7\linewidth]{pictures/Screenshot_2015-08-04-05-37-53.png}
\caption{\label{fig:Screen5}Chatroom.}
\end{figure}
\end{center}

There is a small text box at the bottom of the chatroom that can be used to compose your messages and then when ready, you can click on the send button next to it to send the message to your peer. Like this:

\begin{center}
\begin{figure}[H]
\centering
\includegraphics[width=0.7\linewidth]{pictures/Screenshot_2015-08-04-05-38-27.png}
\caption{\label{fig:Screen6}Chatroom 2.}
\end{figure}
\end{center}

To add new contacts via linphone just go back to the main screen and click on the contacts button. This will take you to the Contact list screen. At the to right position of that screen click on the new contact button and you will be led to the following screen:

\begin{center}
\begin{figure}[H]
\centering
\includegraphics[width=0.7\linewidth]{pictures/Screenshot_2015-08-04-05-39-24.png}
\caption{\label{fig:Screen7}Create New Contact.}
\end{figure}
\end{center}

Now you can fill in the details required in the form and then click on Okay to save the new contact. Like so:

\begin{center}
\begin{figure}[H]
\centering
\includegraphics[width=0.7\linewidth]{pictures/Screenshot_2015-08-04-05-40-26.png}
\caption{\label{fig:Screen8}Create New Contact with details.}
\end{figure}
\end{center}

Your call logs can be found by going back to the main screen and then clicking on the History button at the bottom left screen menu. The log for outgoing and incoming calls are shown here:

\begin{center}
\begin{figure}[H]
\centering
\includegraphics[width=0.7\linewidth]{pictures/Screenshot_2015-08-04-05-42-53.png}
\caption{\label{fig:Screen9}Call Log History.}
\end{figure}
\end{center}

\newpage
\subsection{System Configuration}
\textbf{Troubleshooting, FAQ and Problems} \\

\textbf{Connection problems} \\
I try to phone to my friend <sip:toto@example.com>, but nothing happens, no ring, nothing
at all. \\
You must verify that linphone uses the network interface that connects you to the
internet (or to the network where calls should go). \\
Use the property box, section Network, to select the correct network interface.
In other case, the person you are contacting may be not reachable at the moment... \\

\textbf{Audio problems} \\
Linphone seems to connect to the remote sip url, it rings, but when the callee answers,
nothing happens and we can't hear each other. \\

Most people get problems because they don't choose the correct network interface
in the property box, section network. \\

In other cases, it will fail. \\
First rise up playback and recording level. \\
If the voice is sometines cutted, you can modify parameter RTP\->jitter \\
compensation in the property box to greater values to avoid this. But it increases the
delay transmission. \\

\textbf{FAQ}
How do I use Linphone Calling? \\
Linphone Calling lets you call your friends and family using Linphone for free, even if they're in another country. Currently, Linphone Calling is available on Android. \\
Linphone Calling uses your phone's Internet connection rather than your cellular plan's voice minutes. Data charges may apply.
Please note: you can't access 10111 and other emergency service numbers through Linphone. You must make alternative communication arrangements to make emergency calls. \\

\textbf{Bugs reporting and suggestions} \\
First go to linphone's home page at http://www.linphone.org to check if you have the
latest version of linphone. \\

\end{document}

